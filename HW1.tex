\documentclass[psamsfonts]{amsart}
\usepackage{amssymb,amsfonts}
\usepackage[all,arc]{xy}
\usepackage{enumerate}
\usepackage{mathrsfs}
\usepackage{tikz}
%\usepackage[letterpaper, portrait, margin=in]{geometry}
\usepackage{setspace}
\usepackage{graphicx}
\usepackage{caption}
\usepackage{subcaption}
\usepackage{breqn}
%\onehalfspacing

%theoremstyle{plain} --- default
\newtheorem{thm}{Theorem}[section]
\newtheorem{cor}[thm]{Corollary}
\newtheorem{prop}[thm]{Proposition}
\newtheorem{lem}[thm]{Lemma}
\newtheorem{conj}[thm]{Conjecture}
\newtheorem{quest}[thm]{Question}

\theoremstyle{definition}
\newtheorem{defn}[thm]{Definition}
\newtheorem{defns}[thm]{Definitions}
\newtheorem{con}[thm]{Construction}
\newtheorem{exmp}[thm]{Example}


\newtheorem{exmps}[thm]{Examples}
\newtheorem{notn}[thm]{Notation}
\newtheorem{notns}[thm]{Notations}
\newtheorem{addm}[thm]{Addendum}
\newtheorem{exer}[thm]{Exercise}
\newtheorem{prob}[thm]{Problem}

\theoremstyle{remark}
\newtheorem{rem}[thm]{Remark}
\newtheorem{rems}[thm]{Remarks}
\newtheorem{warn}[thm]{Warning}
\newtheorem{sch}[thm]{Scholium}

\DeclareMathOperator{\sgn}{sgn}

%%%%COMMENT MACROS%%%%%%%%%%%%%%%%%%%%%%%%%%%%%%%%
%%%%COMMENT MACROS%%%%%%%%%%%%%%%%%%%%%%%%%%%%%%%%
%\usepackage{showlabels}
% LABELS!!
%\usepackage{times}%
%\usepackage[T1]{fontenc}%
%\usepackage{mathrsfs}%

%\usepackage[dvips]{graphics}
\usepackage{epsfig}
\usepackage{amsmath}
%\usepackage{hyperref, amsmath, amsthm, amsfonts, amscd, flafter,epsf}
%\usepackage{amsfonts,amsthm, ,amscd}
%\input amssym.def
%\input amssym.tex
\usepackage{color}
\newcommand{\will}[1]{{\color{blue} \sf $\clubsuit\clubsuit\clubsuit$ Will: [#1]}}
\newcommand{\artem}[1]{{\color{green} \sf $\spadesuit\spadesuit\spadesuit$ Artem: [#1]}}
%%%%%END  COMMENT MACROS(April 2011)%%%%%%%%%%%%%%%%%%%
%%%%%END  COMMENT MACROS(April 2011)%%%%%%%%%%%%%%%%%%%

\makeatletter
\let\c@equation\c@thm
\makeatother
\numberwithin{equation}{section}

\bibliographystyle{plain}

\title{Homework 1}

\author{A. Bolshakov}

\date{ \today }

\begin{document}

\maketitle

\section{Problem 1}

Many constraint surfaces can be locally linearized. 
Let us examine the behavior of the dynamical system
\[
\dot{\mathbf{x}} = P_A(\mathbf{x}) - P_B(\mathbf{x})
\]
for this large class of systems by just looking at a simple, 2 dimensional case. 
Let constraint $A$ be
$x_2 = mx_1$
and constraing $B$ be 
$x_2 = -mx_1$
for some slope $m$.
(see Fig. \ref{fig1}).
%If our linear system has different slopes, we can easily get to this situation by rotating and dilating the coordinate axis.
%This is a linear map, so the dynamics are not affected by these transformations.


%Assume we start our dynamical system in the regime 
%$-mx_1 < x_2 < mx_1$ 
%(if this is not satisfied, we can easily rename the axes).


The system is simple enough that we can explicitly solve for $P_A$ and $P_B$:
\[
P_A \left( 
\begin{matrix}
x_1 \\ x_2
\end{matrix}
\right)
= 
\frac{1}{m^2 + 1}
\left(
\begin{matrix}
x_1 + mx_2 \\
mx_1 + m^2x_2
\end{matrix}
\right)
\]
\[
P_B \left( 
\begin{matrix}
x_1 \\ x_2
\end{matrix}
\right)
= 
\frac{1}{m^2 + 1}
\left(
\begin{matrix}
x_1 - mx_2 \\
m^2x_2 - mx_1
\end{matrix}
\right)
\]

We can ignore the $m^2 + 1$ coefficient as we only care about long term behavior.
So, we get
\[
\dot{\mathbf{x}} = 
P_A(\mathbf{x}) - P_B(\mathbf{x}) = 
\left(
\begin{matrix}
2mx_2 \\ 2mx_1
\end{matrix}
\right) 
= 
\left(
\begin{matrix}
0 & 2m \\
2m & 0
\end{matrix}
\right)
\mathbf{x}
\]


This is a classic linear hyperbolic system. The two eigenvalues are $\pm 2m$, so all trajectories except a set of measure $0$ will flow away.


\section{Problem 2}



\section{Problem 3}





\end{document}

